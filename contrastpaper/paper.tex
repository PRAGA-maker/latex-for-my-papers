% main.tex
% Compile: pdflatex main.tex  (or latexmk -pdf main.tex)
% Toggle line numbers:
%   - \reviewtrue  = internal "review copy" with left-margin line numbers
%   - \reviewfalse = arXiv-safe (no line numbers)

\documentclass[11pt]{article}

% ---------- Toggle ----------
\newif\ifreview
\reviewtrue
% \reviewfalse

% ---------- Layout / typography ----------
\usepackage[margin=1in]{geometry}
\usepackage{microtype}
\usepackage{setspace}
\usepackage{newtxtext,newtxmath} % Times-like (common in policy/health/ML drafts)

% ---------- Title / authors ----------
\usepackage{authblk}
\setlength{\affilsep}{0.6em}
\renewcommand\Authfont{\large}
\renewcommand\Affilfont{\normalsize}

% ---------- Line numbers (review copy style) ----------
\usepackage[left]{lineno}
\ifreview
  \linenumbers
  \setpagewiselinenumbers
  \modulolinenumbers[1]
  \renewcommand\linenumberfont{\normalfont\tiny\sffamily}
  \setlength\linenumbersep{10pt}
\fi

% ---------- Misc ----------
\usepackage[hidelinks]{hyperref}

% ---------- Metadata ----------
\title{\textbf{Large Scale Clinical Implications of Contrast Agent Withholding}}
\author[1]{Praneel Patel}
\author[2]{Eric Hong}
\author[3]{Adhityha Guhan}
\author[4]{Michelle Ramim}

\affil[1]{Ohio State University}
\affil[2]{Case Western Reserve University}
\affil[3]{University of Illinois}
\affil[4]{Nova Southeastern University}

\date{December 30, 2025}

\begin{document}
\maketitle

% Optional: slightly tighter spacing under title block
\vspace{-0.5em}

\begin{abstract}
\noindent\textbf{Background:}
Contrast-enhanced imaging improves diagnostic yield, but contrast is often withheld due to concerns about contrast-associated acute kidney injury (CA-AKI) and cautious institutional risk posture. Professional society guidance (e.g., American College of Radiology) has progressively supported IV contrast use even in high-risk patients (acute renal dysfunction, CKD stage 4--5), emphasizing contextual clinical judgment and prophylaxis; however, real-world adherence and the clinical consequences of withholding remain poorly quantified.

\noindent\textbf{Methods:}
Using NIH All of Us EHR data (>630{,}000 participants), we emulated target trials comparing contrast versus no-contrast strategies in four high-volume imaging contexts with protocol equipoise (CT chest, CT pelvis, CT abdomen/pelvis, MRI abdomen; analytic \(n=59{,}573\)). We estimated risk differences for 30-day AKI with cross-fitted efficient influence function augmented inverse probability weighting (EIF-AIPW), overlap trimming, and doubly robust nuisance models.

\noindent\textbf{Results:}
Contrast was avoided in 19\% of eligible encounters. In the highest-risk renal subgroup (baseline eGFR \(<30\) mL/min/1.73 m\(^2\); \(n=358\)), contrast use was associated with a 12.6--percentage-point lower risk of 30-day AKI (49\% relative reduction). Across CT procedures, estimated effects were consistently protective and largest for CT chest (5.6--percentage-point reduction; 39\% relative). MRI abdomen showed a small 0.07--percentage-point increase.

\noindent\textbf{Conclusions:}
In a large national EHR cohort, contrast avoidance in imaging with protocol equipoise was common, and contrast administration was not associated with higher 30-day AKI risk---even among patients with severe renal dysfunction. These findings support re-evaluating renal-risk--motivated contrast avoidance and motivate scalable trial-emulation audits across additional outcomes and decision policies.
\end{abstract}

\end{document}
